
\documentclass{article}
%%%%%%%%%%%%%%%%%%%%%%%%%%%%%%%%%%%%%%%%%%%%%%%%%%%%%%%%%%%%%%%%%%%%%%%%%%%%%%%%%%%%%%%%%%%%%%%%%%%%%%%%%%%%%%%%%%%%%%%%%%%%%%%%%%%%%%%%%%%%%%%%%%%%%%%%%%%%%%%%%%%%%%%%%%%%%%%%%%%%%%%%%%%%%%%%%%%%%%%%%%%%%%%%%%%%%%%%%%%%%%%%%%%%%%%%%%%%%%%%%%%%%%%%%%%%
%TCIDATA{OutputFilter=LATEX.DLL}
%TCIDATA{Version=5.50.0.2953}
%TCIDATA{<META NAME="SaveForMode" CONTENT="1">}
%TCIDATA{BibliographyScheme=Manual}
%TCIDATA{Created=Friday, September 22, 2017 15:14:43}
%TCIDATA{LastRevised=Friday, September 22, 2017 15:59:01}
%TCIDATA{<META NAME="GraphicsSave" CONTENT="32">}
%TCIDATA{<META NAME="DocumentShell" CONTENT="Standard LaTeX\Blank - Standard LaTeX Article">}
%TCIDATA{CSTFile=40 LaTeX article.cst}

\newtheorem{theorem}{Theorem}
\newtheorem{acknowledgement}[theorem]{Acknowledgement}
\newtheorem{algorithm}[theorem]{Algorithm}
\newtheorem{axiom}[theorem]{Axiom}
\newtheorem{case}[theorem]{Case}
\newtheorem{claim}[theorem]{Claim}
\newtheorem{conclusion}[theorem]{Conclusion}
\newtheorem{condition}[theorem]{Condition}
\newtheorem{conjecture}[theorem]{Conjecture}
\newtheorem{corollary}[theorem]{Corollary}
\newtheorem{criterion}[theorem]{Criterion}
\newtheorem{definition}[theorem]{Definition}
\newtheorem{example}[theorem]{Example}
\newtheorem{exercise}[theorem]{Exercise}
\newtheorem{lemma}[theorem]{Lemma}
\newtheorem{notation}[theorem]{Notation}
\newtheorem{problem}[theorem]{Problem}
\newtheorem{proposition}[theorem]{Proposition}
\newtheorem{remark}[theorem]{Remark}
\newtheorem{solution}[theorem]{Solution}
\newtheorem{summary}[theorem]{Summary}
\newenvironment{proof}[1][Proof]{\noindent\textbf{#1.} }{\ \rule{0.5em}{0.5em}}
\input{tcilatex}

\begin{document}


\section{\protect\FRAME{ftbpF}{3.9167in}{1.0923in}{0pt}{}{}{Figure}{\special%
{language "Scientific Word";type "GRAPHIC";maintain-aspect-ratio
TRUE;display "USEDEF";valid_file "T";width 3.9167in;height 1.0923in;depth
0pt;original-width 5.1526in;original-height 1.4166in;cropleft "0";croptop
"1";cropright "1";cropbottom "0";tempfilename
'OWP8YD05.bmp';tempfile-properties "XPR";}}}

\bigskip 

\section{ \ \ \ \ \ \ \ \ \ \ \ \ \ \ \ \ \ \ \ \ \ SISTEMAS OPERATIVOS LINUX%
}

\bigskip 

\section{ \ \ \ \ \ \ \ APLICACI\'{O}N EN ANDROID CON DOS FRAGMENTS}

\bigskip 

\section{ \ \ \ \ \ \ \ \ \ \ \ \ \ \ \ \ \ \ \ \ \ \ JOS\'{E} IGNACIO NAVA
AGUIRRE}

\bigskip 

\section{ \ \ \ \ \ \ \ \ \ \ \ \ \ \ \ \ \ \ \ \ \ \ \ \ \ \ \ \ \ \ \ \ 22
/ SEPT / 2017}

\bigskip 

\bigskip 

\bigskip 

\subsection{\protect\bigskip\ \ \ \ \ \ \ \ \ \ \ \ \ \ \ \ \ \ \ \ \ \ \ \
\ \ \ \ \ \ \ \ \ \ \ \ \ DESCRIPCI\'{O}N:}

\bigskip Un fragment representa un comportamiento o una parte de la interfaz
de usuario en una Activity. Puedes combinar m\'{u}ltiples fragmentos en una
sola actividad para crear una IU multipanel y volver a usar un fragmento en m%
\'{u}ltiples actividades. Puedes pensar en un fragmento como una secci\'{o}n
modular de una actividad que tiene su ciclo de vida propio, recibe sus
propios eventos de entrada y que puedes agregar o quitar mientras la
actividad se est\'{e} ejecutando (algo as\'{\i} como una "subactividad" que
puedes volver a usar en diferentes actividades).

\bigskip 

Un fragmento siempre debe estar integrado a una actividad y el ciclo de vida
del fragmento se ve directamente afectado por el ciclo de vida de la
actividad anfitriona. Por ejemplo, cuando la actividad est\'{a} pausada,
tambi\'{e}n lo est\'{a}n todos sus fragmentos, y cuando la actividad se
destruye, lo mismo ocurre con todos los fragmentos. Sin embargo, mientras
una actividad se est\'{a} ejecutando (est\'{a} en el estado del ciclo de
vida reanudada ), puedes manipular cada fragmento de forma independiente;
por ejemplo, para agregarlos o quitarlos. Cuando realizas una transacci\'{o}%
n de fragmentos como esta, tambi\'{e}n puedes agregarlos a una pila de
actividades administrada por la actividad; cada entrada de la pila de
actividades en la actividad es un registro de la transacci\'{o}n de
fragmentos realizada. La pila de actividades le permite al usuario invertir
una transacci\'{o}n de fragmentos (navegar hacia atr\'{a}s) al presionar el
bot\'{o}n Atr\'{a}s.

\bigskip 

\subsection{ \ \ \ \ \ \ \ \ \ \ \ \ \ \ \ \ \ \ \ \ \ \ \ \ \ IMAGEN DEL
ACTIVITY\_MAIN.XML}

\FRAME{ftbpF}{2.7294in}{4.5299in}{0pt}{}{}{Figure}{\special{language
"Scientific Word";type "GRAPHIC";maintain-aspect-ratio TRUE;display
"USEDEF";valid_file "T";width 2.7294in;height 4.5299in;depth
0pt;original-width 5.1526in;original-height 8.5694in;cropleft "0";croptop
"1";cropright "1";cropbottom "0";tempfilename
'OWP8YD06.bmp';tempfile-properties "XPR";}}

\bigskip 

\bigskip \TEXTsymbol{<}?xml version="1.0" encoding="utf-8"?\TEXTsymbol{>}

\TEXTsymbol{<}LinearLayout
xmlns:android="http://schemas.android.com/apk/res/android"

xmlns:app="http://schemas.android.com/apk/res-auto"

xmlns:tools="http://schemas.android.com/tools"

android:layout\_width="match\_parent"

android:layout\_height="match\_parent"

android:paddingLeft="16dp"

android:paddingRight="16dp"

android:paddingBottom="16dp"

tools:context=".MainActivity"

android:orientation="vertical"\TEXTsymbol{>}

\TEXTsymbol{<}Button

android:layout\_width="250dp"

android:layout\_height="wrap\_content"

android:text="LOAD FIRST FRAGMENT"

android:id="@+id/bn"

android:layout\_gravity="center\_horizontal"

/\TEXTsymbol{>}

\TEXTsymbol{<}RelativeLayout

android:layout\_width="match\_parent"

android:layout\_height="400dp"

android:layout\_marginTop="20dp"

android:id="@+id/fragment\_container"

/\TEXTsymbol{>}

\TEXTsymbol{<}/LinearLayout\TEXTsymbol{>}

\subsection{ \ \ \ \ \ \ \ \ \ \ \ \ \ \ \ \ \ \ \ \ \ \ \ \ \ \ \
FRAGMENT\_ONE\_LAYOUT.XML}

\FRAME{ftbpF}{2.5382in}{4.19in}{0pt}{}{}{Figure}{\special{language
"Scientific Word";type "GRAPHIC";maintain-aspect-ratio TRUE;display
"USEDEF";valid_file "T";width 2.5382in;height 4.19in;depth
0pt;original-width 5.1802in;original-height 8.5694in;cropleft "0";croptop
"1";cropright "1";cropbottom "0";tempfilename
'OWP8YD07.bmp';tempfile-properties "XPR";}}

\bigskip 

\TEXTsymbol{<}FrameLayout
xmlns:android="http://schemas.android.com/apk/res/android"

xmlns:tools="http://schemas.android.com/tools"

android:layout\_width="match\_parent"

android:layout\_height="match\_parent"

tools:context="com.example.uidp2591.fragment.FragmentOne"

android:background="\#077751"\TEXTsymbol{>}

\TEXTsymbol{<}TextView

android:layout\_width="match\_parent"

android:layout\_height="wrap\_content"

android:text="Welcome to fragment One"

android:textAppearance="?android:textAppearanceLarge"

android:layout\_gravity="center"

android:gravity="center"

/\TEXTsymbol{>}

\TEXTsymbol{<}/FrameLayout\TEXTsymbol{>}

\bigskip 

\subsection{\protect\bigskip\ \ \ \ \ \ \ \ \ \ \ \ \ \ \ \ \ \ \ \ \ \ \
FRAGMENT\_TWO\_LAYOUT.XML}

\FRAME{ftbpF}{2.5867in}{4.2739in}{0pt}{}{}{Figure}{\special{language
"Scientific Word";type "GRAPHIC";maintain-aspect-ratio TRUE;display
"USEDEF";valid_file "T";width 2.5867in;height 4.2739in;depth
0pt;original-width 5.1664in;original-height 8.5556in;cropleft "0";croptop
"1";cropright "1";cropbottom "0";tempfilename
'OWP8YD08.bmp';tempfile-properties "XPR";}}

\bigskip 

\TEXTsymbol{<}FrameLayout
xmlns:android="http://schemas.android.com/apk/res/android"

xmlns:tools="http://schemas.android.com/tools"

android:layout\_width="match\_parent"

android:layout\_height="match\_parent"

tools:context="com.example.uidp2591.fragment.FragmentTwo"

android:background="\#8f3a1e"\TEXTsymbol{>}

\TEXTsymbol{<}TextView

android:layout\_width="match\_parent"

android:layout\_height="wrap\_content"

android:text="Welcome to fragment Two"

android:textAppearance="?android:textAppearanceLarge"

android:layout\_gravity="center"

android:gravity="center"

/\TEXTsymbol{>}

\TEXTsymbol{<}/FrameLayout\TEXTsymbol{>}

\bigskip 

\bigskip 

\subsection{\protect\bigskip FragmentTwo.java}

package com.example.uidp2591.fragment;

import android.os.Bundle;

import android.support.v4.app.Fragment;

import android.view.LayoutInflater;

import android.view.View;

import android.view.ViewGroup;

/**

* A simple \{@link Fragment\} subclass.

*/

public class FragmentTwo extends Fragment \{

public FragmentTwo() \{

// Required empty public constructor

\}

@Override

public View onCreateView(LayoutInflater inflater, ViewGroup container,

Bundle savedInstanceState) \{

// Inflate the layout for this fragment

return inflater.inflate(R.layout.fragment\_two\_layout, container, false);

\}

\}

\bigskip 

\subsection{FragmentOne.java}

package com.example.uidp2591.fragment;

import android.os.Bundle;

import android.support.v4.app.Fragment;

import android.view.LayoutInflater;

import android.view.View;

import android.view.ViewGroup;

/**

* A simple \{@link Fragment\} subclass.

*/

public class FragmentOne extends Fragment \{

public FragmentOne() \{

// Required empty public constructor

\}

@Override

public View onCreateView(LayoutInflater inflater, ViewGroup container,

Bundle savedInstanceState) \{

// Inflate the layout for this fragment

return inflater.inflate(R.layout.fragment\_one\_layout, container, false);

\}

\}

\bigskip 

\bigskip 

\subsection{MainActivity.java}

\bigskip 

import android.app.Fragment;

//import android.app.FragmentManager;

//import android.app.FragmentTransaction;

import android.support.v4.app.FragmentManager;

import android.support.v4.app.FragmentTransaction;

import android.support.v7.app.AppCompatActivity;

import android.os.Bundle;

import android.view.View;

import android.widget.Button;

import static com.example.uidp2591.fragment.R.id.fragment\_container;

public class MainActivity extends AppCompatActivity \{

boolean status = false;

Button bn;

@Override

protected void onCreate(Bundle savedInstanceState) \{

super.onCreate(savedInstanceState);

setContentView(R.layout.activity\_main);

bn = (Button)findViewById(R.id.bn);

bn.setOnClickListener(new View.OnClickListener()\{

@Override

public void onClick(View v) \{

FragmentManager fragmentManager = getSupportFragmentManager();

FragmentTransaction fragmentTransaction = fragmentManager.beginTransaction();

if(!status)

\{

FragmentOne f1 = new FragmentOne();

fragmentTransaction.add(R.id.fragment\_container, f1);

fragmentTransaction.addToBackStack(null);

fragmentTransaction.commit();

bn.setText("LOAD SECOND FRAGMENT");

status = true;

\}

else

\{

FragmentTwo f2 = new FragmentTwo();

fragmentTransaction.add(R.id.fragment\_container,f2);

fragmentTransaction.addToBackStack(null);

fragmentTransaction.commit();

bn.setText("LOAD FIRST FRAGMENT");

status = false;

\}

\}

\});

\}

\}

\bigskip 

\bigskip 

\bigskip 

\bigskip 

\bigskip 

\bigskip 

\bigskip 

\bigskip 

\bigskip 

\bigskip 

\bigskip 

\bigskip 

\bigskip 

\end{document}
